%导言区
%\documentclass{article}     %文档类:article,book,report,letter
\documentclass{ctexart} 
%\usepackage{ctex}

\newcommand\degree{^\circ}

\title{\heiti 杂谈勾股定理}   %文档标题
\author{\kaishu 纪帆}    %文章作者
\date{\today}   %文档时间

%正文区(文稿区)
\begin{document}    %环境名称,一个latex只能有一个document环境 
    \maketitle
    勾股定理可以用现代语言表述如下:

    直角三角形斜边的平方等于两腰的平方和。

    可以用符号语言表述为:设直角三角形 $\triangle ABC$,其中 $\angle C=90\degree$,则有:
    \begin{equation}    %给公式加编号
        AB^2=BC^2+AC^2
    \end{equation}
\end{document}